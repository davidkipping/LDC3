%% INTRODUCTION
%%

Stellar limb darkening affects a wide range of astronomical observations, such
as exoplanetary transits \citep{mandel:2002}, microlensed light curves 
(e.g. \citealt{witt:1995}; \citealt{zub:2011}), rotational modulations 
\citep{macula:2012}, ellipsoidal variations \citep{morris:1993}, interferometric 
images of stars (e.g. \citealt{aufdenberg:1995}) and eclipsing binaries 
\citep{kopal:1950}. When interpreting such observations, the assumed shape of 
the limb darkening intensity profile may significantly affect the inferred 
parameters of interest \citep{csizmadia:2013}. Consequently, an accurate 
treatment of limb darkening is crucial, even when the effect itself is a 
``nuisance'' phenomenon.

In many practical cases, limb darkening is treated by describing the stellar 
intensity profile with a closed-form analytic model. This model is usually 
called the ``limb darkening law'', which is designed to provide an accurate 
analytic approximation the true profile. The major advantage of modeling the
intensity profile analytically is that the astronomical phenomena under
investigation may also be modeled analytically, offering computational
expedience and greater physical insight. As an example, in the field of 
exoplanet transits, the shape of a transit light curve can be described with 
a closed-form, analytic model under the assumption of a polynomial-based 
description of the stellar intensity profile \citep{mandel:2002,gimenez:2006}.

All of the commonly used limb darkening laws may be described as a summation of
one or more simple functions, each of which is weighted by a so-called limb
darkening coefficient (LDC). For example, the popular quadratic limb darkening 
law describes the intensity profile as a quadratic series expansion with respect 
to the angle between the line of sight and the emergent intensity ($\mu$) and
has two LDCs \citep{kopal:1950}. These LDCs control the shape of the intensity 
profile, subject to the flexibility granted by the limb darkening law's 
complexity. It is therefore necessary to make at least two major decisions 
with how to treat limb darkening; what limb darkening law should be used and 
what LDCs should be assigned to this law?

A typical approach for assigning LDCs is to adopt a set of so-called 
``theoretical'' LDCs. In this framework, one first simulates an intensity 
profile using a sophisticated stellar atmosphere model at a particular 
wavelength or over a chosen integrated bandpass. One then takes this 
simulation and regresses the limb darkening law of choice to it by finding
the maximum likelihood set of LDCs (in some instances the LDCs are restricted
to ensure physically sound profiles). Since the simulated profile is sensitive 
to several parameters defining the stellar surface (e.g. effective temperature, 
metallicity, surface gravity), several groups have produced grid tabulations of 
LDCs for a wide range of plausible inputs and assumed limb darkening law (e.g.
see \citealt{vanhamme:1993}, \citealt{diaz:1995}, \citealt{claret:2000} and
\citealt{sing:2010}).

In recent years, there has been a shift away from using theoretical LDCs in 
favour of freely fitting LDCs simultaneous to the exploration of the other 
parameters of interest (e.g. \citealt{knutson:2007}, \citealt{kipbak:2011a,
kipbak:2011b} and \citealt{kreidberg:2014}). This approach, enabled by advances 
in modern computers, allows one to propagate our ignorance regarding the shape 
of the intensity profile into the estimation of the parameters of interest, 
leading to more accurate estimates parameter uncertainties. Such a procedure
also allows one to decouple from any possible errors in the stellar atmosphere
models themselves, which have been shown to sometimes be inconsistent with
observed limb darkening profiles \citep{howarth:2011,epinoza:2015}.

%In the case where the star is well-characterized and theoretical models are 
%highly trusted, the LDCs may be drawn from an informative prior, computed via 
%Monte Carlo simulation of the stellar atmosphere models (e.g. 
%\citealt{berta:2012} and \citealt{kepler22:2013}). In the case where the 
%star is poorly characterized and/or the theoretical models are not trusted, an 
%uninformative prior is typically adopted.

One major challenge when attempting to freely fit LDCs is that many combinations
of the LDCs are unphysical. For example, a specific choice of LDCs may result
in the intensity profile being occasionally negative. These regions of 
unphysical parameter space may coincide with likelihood minima, leading to
erroneous parameter posterior distributions. Softer, extended likelihood
minima (typical of low signal-to-noise data) may spill over into the unphysical
parameter space, causing the final parameter posteriors to be marginalized over 
large swaths of unphysical parameter space leading to unnecessarily swollen 
error bars.

There are two solutions to this problem. The first is to define a set of 
criteria which quickly allow us to identify forbidden combinations of the LDCs.
Armed with such criteria, one could then test each realization and accept or
reject the realization accordingly. This works fine unless the volume of
forbidden parameter space is large, in which case one will severely impede
a regression algorithm's efficiency since most trials are being rejected
as unphysical. A more elegant solution is to directly sample exclusively (or
at least efficiently) from the physically allowed volume of parameter space (and 
also in such a way that the LDCs can be uniformly, or nearly uniformly, 
sampled). This provides the benefits of a very high efficiency, physically sound 
priors and no need for testing criteria at each realization. However, this 
approach comes at the cost of the initial intellectual investment to actually 
solve how to perform efficient sampling in the first place.

Because direct sampling of physical LDCs is challenging, the most complex limb 
darkening law for which this feat has yet been achieved is the simple quadratic
law, where the intensity profile of the star is given by

\begin{align}
I(\mu)/I(1) &= 1 - u_1 (1-\mu) - u_2 (1-\mu)^2,
\label{eqn:Iquad}
\end{align}

where $I(1)$ is the specific intensity at the centre of the disc, $u_1$ and
$u_2$ are the quadratic LDCs and $\mu$ is the cosine of the angle between the 
line of sight and the emergent intensity. In \citet{LD:2013}, we showed that
physically sound $u_1$ and $u_2$ LDCs reside within a triangle on the
$u_1$-$u_2$ plane, from which uniform samples can be easily drawn by
re-parameterizing to $q_1 = (u_1+u_2)^2$ and $q_2 = 0.5u_1(u_1+u_2)^{-1}$.
Other simple two-parameter laws were considered as well in this work.

Despite this successes with two-parameter laws, these laws are fundamentally
limited in their ability to accurately describe realistic limb darkening 
profiles. This translates to systematic uncertainty in any and all model 
parameters which are covariant with the limb darkening properties. This may,
for example, limit our ability to accurately measure the radius of a transiting
exoplanet and thus make inferences of its composition. We are therefore motivated
to extend the \citet{LD:2013} analysis to a more sophisticated limb darkening
law.

The most accurate closed-form limb darkening law is the \citet{claret:2000} 
four-parameter non-linear law, described by,

\begin{align}
I(\mu)/I(1) = 1 &- c_1 (1-\mu^{1/2}) - c_2 (1-\mu) \nonumber\\
\qquad&- c_3 (1-\mu^{3/2}) - c_4 (1-\mu^{2}),
\label{eqn:Iclaret}
\end{align}

In numerous independent studies, this law has been found to provide the most 
accurate description of simulated intensity profiles (e.g. 
\citealt{claret:2000}, \citealt{sing:2010} and \citealt{magic:2015}) versus 
competing models. This is, however, not surprising since this law also utilizes 
the greatest number of LDCs. Analytically identifying the unphysical 
combinations of these four LDCs remains an outstanding and formidable challenge. 
A more tractable problem that we consider in this work is the allowed volume of 
LDCs (and methods to directly sample from said volume) in the case of the 
\citet{sing:2009} three-parameter law:

\begin{align}
I(\mu)/I(1) &= 1 - c_2 (1-\mu) - c_3 (1-\mu^{3/2}) - c_4 (1-\mu^{2}),
\label{eqn:Ising}
\end{align}

\citet{sing:2010} argue that dropping the $c_1$ term is motivated by
solar data \citep{neckel:1994} and 3D stellar models \citep{bigot:2006}, which
show that $I(\mu)$ varies smoothly at small $\mu$, meaning that a $\mu^{1/2}$
term is superfluous. We therefore argue that the \citet{sing:2009} law
offers both a significant improvement in accuracy and yet is simple enough for
us to analytically constrain the allowed LDCs.
