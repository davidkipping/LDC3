%% CRITERIA
%%

\subsection{Physical Conditions}
\label{sub:physicalconditions}

We define the following \textit{physical conditions} with respect to a limb
darkened stellar intensity profile:

\begin{itemize}
\item[{\textbf{(I)}}] an everywhere-positive intensity profile,
\item[{\textbf{(II)}}] a monotonically decreasing intensity profile from the
centre of the star to the limb,
\item[{\textbf{(III)}}] the intensity profile has a negative curl at the limb.
\end{itemize}

Physical conditions \I\ and \II\ are the same two imposed in our previous paper,
\citet{LD:2013}. Physical condition \III\ is motivated by the expectation that
the intensity rapidly drops off towards the limb (and is discussed in more 
detail later in \S\ref{sub:criterionE}). Throughout this work, we refer to a set 
of LDCs satisfying these three conditions as being physical, and LDCs otherwise 
are defined as unphysical. In what follows, we explore the consequences of these 
simple constraints.

\subsection{Criterion A}

Physical condition \I\ demands that $I(\mu)>0$ $\forall$ $0\leq\mu<1$. We begin 
by evaluating this condition at two extrema of $\mu\to1$ and $\mu\to0$, in a 
similar manner to the approach adopted in \citet{LD:2013}:

\begin{align}
\lim_{\mu\to1} I &= 1 > 0,\nonumber \\
\lim_{\mu\to0} I &= 1 - c_2 - c_3 -  c_4 > 0.
\end{align}

The upper line clearly has no constraining power, but the second line provides
our first criterion of

\begin{align}
c_2 + c_3 + c_4 < 1.
\label{eqn:criterionA}
\end{align}

\subsection{Criteria B \& C}

Next, we consider physical condition \II, which demands that 
$\partial I/\partial \mu>0$ $\forall$ $0\leq\mu<1$:

\begin{align}
\frac{ \partial I(\mu) }{\partial \mu } &= c_2 + \frac{3}{2} c_3 \mu^{1/2} + 2 c_4 \mu.
\end{align}

As was done in the previous subsection, let us evaluate the above in the extreme 
cases of $\mu\to1$ and $\mu\to0$, yielding

\begin{align}
\lim_{\mu\to1} \frac{ \partial I(\mu) }{\partial \mu } &= c_2 + \frac{3}{2} c_3 + 2 c_4 > 0,\nonumber\\
\lim_{\mu\to0} \frac{ \partial I(\mu) }{\partial \mu } &= c_2 > 0.
\end{align}

These two expressions provide our criteria B and C, which are respectively
given by

\begin{align}
2 c_2 + 3 c_3 + 4 c_4 > 0,
\label{eqn:criterionB}
\end{align}

and

\begin{align}
c_2 > 0.
\label{eqn:criterionC}
\end{align}

\subsection{Criterion D}

Physical condition \II\ tells us that the intensity decreases from the centre of 
the star to the limb. This implies that the intensity everywhere (except at the
centre of the star) is less than that present at the centre of the star, or
mathematically that

\begin{align}
I(\mu) < \lim_{\mu\to1} I(\mu).
\end{align}

One simple closed-form result from this constraint occurs by comparing the
intensity at the limb to the centre via:

\begin{align}
\lim_{\mu\to0} I(\mu) < \lim_{\mu\to1} I(\mu).
\end{align}

This condition, derived by physical condition \II, provides our fourth
criterion,

\begin{align}
c_2 + c_3 + c_4 > 0.
\label{eqn:criterionD}
\end{align}

\subsection{Criterion E}
\label{sub:criterionE}

Consider the behaviour of the intensity profile at the limb. By virtue of
condition \II, the intensity profile must be decreasing as we approach the 
boundary. We therefore expect a negative gradient with respect to $r$, or 
equivalently a positive gradient with respect to $\mu$, since 
$\partial r/\partial \mu$ is always negative. We also expect that at the limb 
the gradient of the gradient (i.e. the curl) is negative. This is consistent 
with the asymptotic-like behaviour expected due to foreshortening near the limb, 
causing the gradient to become ever-more negative and defines physical condition 
\III. Note that a negative curl with respect to $r$ is equivalent to a negative 
curl with respect to $\mu$, since we now multiply by $(\partial r/\partial \mu)$ 
twice, leading to a double negative. The curl may be expressed as

\begin{align}
\frac{\partial^2 I(\mu)}{\partial \mu^2} = \frac{3}{4} c_3 + 2 c_4 \mu^{1/2}.
\end{align}

At the limb then ($\mu \to 0$), we expect that

\begin{align}
\lim_{\mu\to0} \Big(\frac{\partial^2 I(\mu)}{\partial \mu^2}\Big) < 0,
\end{align}

which defines criterion E,

\begin{align}
c_3 < 0.
\label{eqn:criterionE}
\end{align}

\subsection{Criterion F}
\label{sub:criterionF}

Physical condition \I\ requires that $I(\mu)$ is everywhere positive. Combining
\I\ with \II\ implies that $I(\mu)$ must be less than unity everywhere, which
one may consider to be physical condition \textbf{I'}. Writing this out along 
with \II, one may show that

\begin{align}
-c_2 \mu - c_3 \mu^{3/2} - c_4 \mu^2 > -c_2 - c_3 - c_4,\\
\frac{2}{3} c_2 \mu + c_3 \mu^{3/2} + \frac{4}{3} c_4 \mu^2 > 0.
\end{align}

Adding the two inequalities shown above cancels out the $c_3 \mu^{3/2}$ terms
and leaves us with a quadratic equation:

\begin{align}
c_4 \mu^2 - c_2 \mu > -3(c_2 + c_3 + c_4).
\end{align}

From Criterion A, we know that the sum of the coefficients must be less than 
unity, implying that in the limit of $\mu\to1$, we have,

\begin{align}
c_4 - c_2 > -3.
\label{eqn:c4c2diff}
\end{align}

Starting from criterion B and invoking criterion E, we can also show that:

\begin{align}
2c_2 + 3c_3 + 4c_4 > 0,\nonumber\\
3c_3 > -2c_2 - 4c_4,\nonumber\\
-2c_2 - 4c_4 < 3c_3 < 0,\nonumber\\
c_2 + 2c_4 > 0.
\label{eqn:c4c2sum}
\end{align}

Summing Equations~\ref{eqn:c4c2diff} \& \ref{eqn:c4c2sum} together yields

\begin{align}
c_4 > -1.
\end{align}

Through numerical experimentation, we find that applying the slightly more 
conservative bound of $c_4>0$ yields a more symmetric loci of allowed points
(as discussed later in \S\ref{sec:transformations}), from which it is easier to 
directly sample. We therefore modify criterion F to,

\begin{align}
c_4 > 0.
\label{eqn:criterionF}
\end{align}

\subsection{Criterion G}
\label{sub:criterionG}

For our final criterion, we begin by considering physical condition \II:

\begin{align}
\frac{ \partial I(\mu) }{\partial \mu } &> 0,\nonumber\\
2 c_4 \mu + \frac{3}{2} c_3 \mu^{1/2} + c_2 &> 0.
\end{align}

The gradient expressed must be everywhere-positive and so let us compute
the minimum gradient possible, which occurs when the curl equals zero, or
when

\begin{align}
\frac{\partial (2 c_4 \mu + \frac{3}{2} c_3 \mu^{1/2} + c_2) }{ \partial \mu } &= 0,\nonumber\\
\frac{3}{4} c_3 \mu^{-1/2} + 2 c_4 &= 0.
\end{align}

Therefore, the minimum gradient occurs when $\mu=\mu_{\mathrm{min}}$, where we 
define

\begin{align}
\mu_{\mathrm{min}}^{1/2} = -\frac{3c_3}{8c_4}.
\end{align}

In the case where criterion E and F are in effect, then $c_3$ is negative and 
$c_4$ is positive meaning that $\mu_{\mathrm{min}}$ is a real number.

The point $\mu_{\mathrm{min}}$ may or may not be within the range $0<\mu<1$. If 
indeed it is, then implicitly $-1 < 3 c_3/(8 c_4) < 0$ and we require that 
the gradient at this point is positive, giving

\begin{align}
\mathrm{if}\,\,\,-1 < \frac{3 c_3}{8 c_4} < 0\,\mathrm{\,\,\,then}\nonumber\\
c_2 > \frac{9 c_3^2}{32 c_4}.
\end{align}

As with criterion F, we find through numerical tests in 
\S\ref{sec:transformations} that the loci of points can be made symmetric if we 
impose criterion G under \textit{all} circumstances, not just when 
$-1 < 3 c_3/(8 c_4) < 0$. We therefore modify criterion G to

\begin{align}
c_2 > \frac{9 c_3^2}{32 c_4}.
\label{eqn:criterionG}
\end{align}

\subsection{Summary of Analytic Criteria}
\label{sub:summaryofcriteria}

To summarize, our seven analytic criteria on the three LDCs are

\begin{align}
c_2 + c_3 + c_4 &< 1,\,\,\,\mathbf{[A]}\nonumber\\
2 c_2 + 3 c_3 + 4 c_4 &> 0,\,\,\,\mathbf{[B]}\nonumber\\
c_2 &> 0, \,\,\,\mathbf{[C]}\nonumber\\
c_2 + c_3 + c_4 &> 0, \,\,\,\mathbf{[D]}\nonumber\\
c_3 &< 0, \,\,\,\mathbf{[E]}\nonumber\\
c_4 &> 0, \,\,\,\mathbf{[F]}\nonumber\\
c_2 &> \frac{9 c_3^2}{32 c_4}. \,\,\,\mathbf{[G]}
\label{eqn:ALLcriteria}
\end{align}

Using the three physical conditions only, we note that criterion F should
strictly be $c_4>-1$. We have modified this criterion to be slightly more 
conservative so that the loci of allowed LDCs can be transformed into a 
symmetric cone shape depicted later in \S\ref{sec:transformations}. Similarly, 
criterion G strictly only applies when $-1 < 3 c_3/(8 c_4) < 0$ if one uses 
the three physical conditions. We again modify the criterion such that it 
applies under all circumstances, in order to yield a more symmetric loci, as
shown later in \S\ref{sec:transformations}. We provide Python and Fortran code
(\LDC) to test whether these criteria hold, for which the user can also use
the unmodified versions of the criteria if desired (available at \link).

In \S\ref{sec:criteria_tests}, we explore the consequences of these 
modifications and perform numerical tests demonstrating the effectiveness of the 
seven criteria. First though, we calculate the allowed maxima/minima on each
LDC using the seven criteria, as shown in the next section, \S\ref{sec:bounds}.
