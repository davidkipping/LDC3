%% DISCUSSION
%%

In this work, we have presented a set of seven analytic criteria which may be
used to assess the physicality of LDCs associated with the \citet{sing:2009}
three-parameter non-linear limb darkening law. We imposed simple conditions that 
the flux is everywhere positive, monotonically decreases from centre to limb and 
has a negative curl at the limb. Through numerical testing, we have shown that 
points naively sampled with a simple accept/reject algorithm applied to our 
criteria are always physically valid. Additionally, over 95\% of the physically 
allowed loci of LDCs (found through brute force numerical exploration) satisfy 
the seven criteria, demonstrating a very high completeness. Using an unmodified
set of criteria which retains the asymmetries present in the loci of allowed
LDCs, the completeness is 100\%.

Armed with these criteria, we have re-parameterized the LDCs such that the loci 
of allowed points morphologically resembles a regular geometric shape, 
specifically a cone. We have shown that uniformly sampling points from the conal
region in the re-parameterized space yields physically plausible and
uniformly distributed LDCs to high accuracy. Specifically, we find a validity
of 97.4\% and a completeness of 94.4\%. 

Sampling from the conal region may be achieved by drawing a uniform random 
variate in the transformed space $\{\alpha_h,\alpha_r,\alpha_{\theta}\}$, for 
which the relational expressions to the original $\{c_2,c_3,c_4\}$ LDCs are 
provided in Equation~\ref{eqn:alpha}. We also provide public code (\LDC) in 
Python and Fortran to perform both the forward and inverse calculation
between the parameterizations (\link).

Our work provides, for the first time, a practical and efficient framework for 
fitting astronomical data affected by limb darkening with a law supporting three
degrees of freedom. Until now, one had to limit oneself to efficient sampling 
of a two-parameter limb darkening law \citep{LD:2013} and go without the major
improvement in accuracy provided by a three-parameter law, such as that of
\citet{sing:2009}. Alternatively, one would have had to explore and marginalize
over unphysical combinations of LDCs (which we estimate would occur for at 
least 99.9\% of naively sampled points) or numerically test the physicality
of each realization of LDCs (again with an overhead of rejecting the vast
majority of points). In any case, we argue that our solution provides major
advantages and enables the community to practically fit more complex limb 
darkening profiles for the first time.

The only published grids of theoretical LDCs using the \citet{sing:2009} law
comes from \citet{sing:2010}. With the Kepler bandpass, we find that 99.6\% of 
the \citet{sing:2010} tabulated points satisfy physical conditions \I, \II\ and 
\III. Further more, 97.7\% of these tabulated points reproduce an $\boldalpha$
transformed LDC within the unit cube. These values again demonstrate that
the $\boldalpha$ parameterization can be practically used to explore the
physically allowed LDCs. 

In some applications, having ``only'' $97.3$\% of the LDCs being physically 
valid may be insufficient and one may wish to ensure 100\% validity. Since 
the seven analytic criteria guarantee 100\% validity, one may draw a set of LDCs 
using the $\boldalpha$ parameterization and then test if this realization 
satisfies the seven criteria. In practice, criteria B and D are never
violated by points sampled from the conal region and thus it is only necessary
to test five criteria. This approach enables a guaranteed physically plausible 
set of LDCs at minimal computational expense. To aid the community, our code 
\LDC\ can perform this test (\link).

Whilst the quadratic law (and other two parameter laws) will likely remain 
suitable for many studies, the analysis of high precision data increasingly 
demands a more sophisticated treatment of limb darkening to avoid this
issue becoming a bottleneck in obtainable accuracy \citep{epinoza:2015}. By 
freely fitting high precision data with our $\boldalpha$ parameterization of 
\citet{sing:2009} limb darkening, one can have greater confidence that the 
parameters of interest are marginalized exclusively over the physically 
plausible parameter space and limb darkening is modeled in a manner more 
consistent with simulations from modern stellar atmosphere models. We also note 
that informative priors on our $\boldalpha$ parameterization may be used as 
well, in cases where one has strong belief in the results of stellar atmosphere 
models and the star is well-characterized already, or alternatively from 
previous posteriors derived from freely fitting the LDCs. For either
informative or uninformative sampling, the $\boldalpha$ parameterization offers
an efficient and physically sound pathway to exploring parameter space when
modeling limb darkening under the three-parameter law.
